% This is a template for BU-ECE Technical Report.
%
% Depending on report content and author preference, a BU-ECE report may be
% in one of the two following styles:
%
%   - genuine report based on ``report'' style, i.e., with chapters, much like
%     a thesis; can be single- or double-sided,
%
%   - report based on ``article'' style, i.e., with no chapters (only sections,
%     subsections, etc.), much like a journal or conference paper; can be
%     single- or double-sided.

% =====================================================================

%\documentclass[12pt]{report}          %Single-sided report style (chapters)
%\documentclass[12pt,twoside]{report}  %Double-sided report style (chapters)
%\documentclass[12pt]{article}         %Single-sided article style (no chapters)
\documentclass[12pt,twoside]{article} %Double-sided article style (no chapters)

\usepackage{bu_ece_report}

% In case an adjustment of vertical or horizontal margins is needed
% due to particular LaTeX/dvips or OS installation, you can uncomment
% and edit the following definitions.
% -------------------------------------------------------------------
%\topmargin       0.00 in
%\oddsidemargin   0.50 in
%\evensidemargin  0.00 in

\begin{document}

% Definitions.
% ------------
\buecedefinitions%
        {Room Occupancy sensing using a thermal tripwire}
        {Occusense: Intermediate Report}
        {Janis Intoy and Emily Lam}
        {April 10, 2017}
        {2017-??} % Number of the report (four year digits and number) What is the number supposed to be????????

% Box with title to fit the opening in the cover
% (adds an empty page in double-sided printing mode).
% ---------------------------------------------------
\buecereporttitleboxpage

% Title page
% (adds an empty page in double-sided printing mode).
% ---------------------------------------------------
\buecereporttitlepage

% Special page, e.g., if the report is restricted or
% to whom it is dedicated, etc., otherwise skip.
% (adds an empty page in double-sided printing mode).
% ---------------------------------------------------
\bueceprefacepage{Here comes a special preface page. For example, if the report
is restricted, then a suitable note can be included. This page can also be used
to indicate to whom the document is dedicated, etc.}

% Report summary; max. 1 page.
% (adds an empty page in double-sided printing mode).
% ---------------------------------------------------
\pagenumbering{roman}
\setcounter{page}{1}
\buecereportsummary{
The efficiency of HVAC (Heating Ventilation \& Air Conditioning) systems can be improved by making them
adaptive to the number of people in a room. Automatic adjustments to room ventilation and temperature based on 
room occupancy reduces energy use and is more cost efficient.
In order to a estimate a room's occupancy level, the Occusense Senior Design team has created
a reliable, thermal sensor system to capture the motion of people through doorways. We aim to develop
a reliable, real-time algorithm to detect the direction of motion of people passing through in order to track 
the number of people in the room.
}

% Table of contents, list of figures and list of tables.
% ``\bueceemptypage'' adds empty page in double-sided
% printing mode and performs ``\clearpage'' in single-sided
% mode.
% ------------------------------------------------------
\tableofcontents\bueceemptypage
\listoffigures\bueceemptypage
\listoftables\bueceemptypage

% Switch on running headers for the report:
%   odd pages  - title (lowercase); if too long, use
%                the first few words followed by ``...'',
%   even pages - last names of the authors.
% -------------------------------------------------------
\buecereportheaders

% Introduction.
% -------------
\pagenumbering{arabic}
\setcounter{page}{1}

\section{Introduction}  % Article style
%\chapter{Introduction}  % Report style
In modern efficient buildings, controlling the HVAC system preemptively can be a huge cost saver. Therefore, a senior design team at Boston University is working towards developing sensing technology and algorithms to count the number of people in a room so that the system is always conscious of room occupancy and can adjust system parameters, such as ventilation, accordingly before feedback sensing technology can detect abnormalities, such as rising temperatures of a crowded room. There are a number of ways to detect occupancy. However, this project assumes that if a room has a low number of entry points, counting people can be done at the entries based on who is entering or leaving the room. As such, there would be a continuous knowledge of the number of people in a room. The senior design team has implemented a privacy-inclined low resolution thermal sensor with a field of view of 30�x120� positioned at the top of the door frame and looking perpendicularly down. It is capable of capturing a 16x4 pixel array at frame rates of 0.5 to 512 Hz. Using this information, our project is tasked at developing algorithms to count the number of people entering or leaving a room. Specifically, this algorithm will
1) detect the presence of a moving person in the frame and 2) determine the direction of motion.

Here comes the introduction and reference to a paper \cite{Mchugh09}.

% Following sections, subsections, etc.
% -------------------------------------
\section{Literature Review}  % Article style
%\chapter{Starting chapter}  % Report style



\section{Problem Statement}  % Article style
%\section{Early section}  % Report style

And this is the first section of this chapter.

\section{Implementation}  % Article style
%\chapter{Another chapter}  % Report style

\subsection{}  % Article style
%\section{New section}  % Report style

Section goes here ...

\section{Experimental Results}  % Article style
%\chapter{Final chapter}  % Report style

\subsection{Another section}  % Article style
%\section{Another section}  % Report style

Section with a figure (Fig.~\ref{fig:example}).

\section{Conclusions}

% Plots (PostScript files) are included through the ``figure'' environment.
% For more complicated figures use the minipage commaned (see LaTeX manual).
% --------------------------------------------------------------------------
\begin{figure}[htb]
%
  \begin{minipage}[t]{0.49\linewidth}\centering
%    \centerline{\epsfig{figure=figures/regbsdcod.eps,width=8.0cm}}
    \Ovalbox{\vbox to 1.5in{\vfill\hbox{\vtop{\hsize=2.5in\hfill}\hfill}\vfill}}
    \medskip
    \centerline{(a)}
  \end{minipage}\hfill
%
  \begin{minipage}[t]{0.49\linewidth}\centering
%    \centerline{\epsfig{figure=figures/regbsdcod.eps,width=8.0cm}}
    \Ovalbox{\vbox to 1.5in{\vfill\hbox{\vtop{\hsize=2.5in\hfill}\hfill}\vfill}}
    \medskip
    \centerline{(b)}
  \end{minipage}

  \bigskip

  \begin{minipage}[t]{0.49\linewidth}\centering
%    \centerline{\epsfig{figure=figures/regbsdcod.eps,width=8.0cm}}
    \Ovalbox{\vbox to 1.5in{\vfill\hbox{\vtop{\hsize=2.5in\hfill}\hfill}\vfill}}
    \medskip
    \centerline{(c)}
  \end{minipage}\hfill
%
  \begin{minipage}[t]{0.49\linewidth}\centering
%    \centerline{\epsfig{figure=figures/regbsdcod.eps,width=8.0cm}}
    \Ovalbox{\vbox to 1.5in{\vfill\hbox{\vtop{\hsize=2.5in\hfill}\hfill}\vfill}}
    \medskip
    \centerline{(d)}
  \end{minipage}
%
  \caption{Block diagram: (a) one; (b) two; (c) three, and (d) four.}
  \label{fig:example}
\end{figure}

% Bibliography.
% -------------
\parskip=0pt
\parsep=0pt
\bibliographystyle{ieeetrsrt}

% Important: substitute your BiBTeX (*.bib) files below.
% ------------------------------------------------------
\bibliography{strings,konrad,manuals}

\end{document}
